En esta tesis se desarrolla el concepto del Transporte de Jets motivado por los estudios de Daniel Pérez-Palau en \cite{Perez2013, Perez2015} y se aplica en casos particulares del problema circular de tres cuerpos. El Transporte de Jets integra (o transporta) el flujo de una ecuación diferencial alrededor de una vecindad (o Jet) $\U$ en lugar de una única  condición inicial $\xo$. Para hacerlo, se parametriza a $\U$ con un polinomio de $k$ variables, donde $k$ es la dimensión del espacio y se aplica a cualquier método de integración numérica, que en este trabajo será el método de Taylor por cuestiones de precisión. Como el método de Taylor depende de $\mathbf{x}_n$ para obtener a $\mathbf{x}_{n+1}$, el Transporte de Jets dependerá de $\mathcal{U}_{\mathbf{x}_n}$ para obtener $\mathcal{U}_{\mathbf{x}_{n+1}}$ y como $\mathcal{U}_{\mathbf{x}_n}$ es un polinomio, operar con éste requiere de la implementación de un álgebra polinomial, que estará guiada por la construcción de Haro en \cite{Haro2009}. Para obtener la vecindad $\mathcal{U}_{\mathbf{x}_n}$ de $\mathbf{x}_n$ basta con evaluar al polinomio en cuestión. Tener parametrizadas las variaciones de la vecindad permite su manipulación antes de evaluarla. Por ello, se proponen en la tesis varios indicadores y algoritmos que aprovechan dicha parametrización para obtener información acerca del sistema de ecuaciones diferenciales. Algunos de estos indicadores ya fueron estudiados por Daniel Pérez en sus artículos, así como los campos escalares dados por el tamaño máximo de las vecindades o el de máxima expansión y contracción de puntos del espacio fase. Por otro lado, él ha hecho estudios de probabilidad de colisión entre satélites y la Tierra. Esta tesis explora dichos resultados y extiende un poco el análisis. Adicionalmente, nuevos indicadores son propuestos y puestos a prueba. Uno involucra la variación de parámetros de un sistema de ecuaciones diferenciales y otro prueba numéricamente la conservación simpléctica de ecuaciones hamiltonianas vía el Transporte de Jets.


En el problema circular de tres cuerpos se aplican dichos indicadores y algoritmos específicamente para encontrar la probabilidad de colisión entre asteroides de diferente radio en el sistema Tierra-Luna, para probar en un sentido numérico que el integrador preserva la simplecticidad del sistema, para estudiar la sensibilidad de condiciones iniciales y para analizar la variación del parámetro de masa alrededor de puntos de equilibrio del espacio de configuraciones.  

Adicionalmente, se evalúa la eficiencia, la presición y los tiempos involucrados para los algoritmos desarrollados, los cuales están disponibles en \href{https://github.com/blas-ko/tesis}{https://github.com/blas-ko/tesis} y se programaron en el lenguaje Julia, haciendo uso de los paquetes \textsf{TaylorSeries} \cite{TaylorSeries} y \textsf{TaylorIntegration} \cite{TaylorIntegration}. 
