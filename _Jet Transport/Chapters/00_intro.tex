Las leyes que gobiernan a la Física pueden ser casi siempre presentadas como un conjunto de ecuaciones diferenciales que describen un sistema. Pensemos en la ecuación de Shrödinger de la Mecánica Cuántica, las ecuaciones de Maxwell del Electromagnetismo, las ecuaciones de Newton y las de Hamilton de la Mecánica Clásica, o las leyes de la Termodinámica en su forma diferencial, por mencionar algunas. Casi siempre se han hecho modelos simplificados y aproximaciones de la teoría para resolver preguntas específicas sobre ésta. Un ejemplo es suponer que la moléculas de un gas diluido no interactúan entre sí, lo cual permite resolver analíticamente la ecuación de estado del gas ideal, que recibe el nombre de ideal ya que no existe gas alguno que cumpla tal suposición. Considero dos principales razones por las cuales se hace ésto. La primera es para entender. Si las suposiciones son simples y la cadena lógica que nos lleva a los resultados es clara, es mucho más fácil generar intuición sobre un tema y nos permite extrapolar a casos más complicados una vez que se entiende el modelo base. En el ejemplo del gas ideal uno genera intuición sobre cómo la física estadística explica las propiedades macroscópicas como la presión, el volumen o la temperatura y, una vez que se adquiere cierta experiencia al respecto, se suelen hacer suposiciones más complejas y/o realistas como el modelo de Van der Wals, que sí toma en consideración la interacción entre las moléculas de un fluido y predice la transición de fase de éste. La segunda razón, y que va de la mano con el objetivo principal de esta tesis, es que resolver ecuaciones diferenciales de manera analítica resulta difícil en general. Nadie ha podido resolver de manera analítica el sistema de ecuaciones diferenciales que describen al problema restringido de tres cuerpos, cuyo estudio lleva más de 300 años. La dificultad de dichas ecucaciones a impulsado a analizarlas sin tener que resolverlas explícitamente. Nacen herramientas como el estudio de órbitas periódicas, puntos singulares, puntos de equilibrio, análisis de estabilidad, entre otras. Éstas no solucionan al problema, pero ayudan a entender cómo funcionan ciertos aspectos dinámicos. Sin embargo, hoy tenemos métodos para encontrar dichas soluciones sin resolverlas de forma analítica con el precio de que no son exactas del todo; éstos se conocen como métodos numéricos y métodos cualitativos.

Los métodos numéricos, a los cuales les prestaremos principal atención en este trabajo, encuentran la solución aproximada\footnote{Que las soluciones sean aproximadas no quiere decir que sean malas. Existen métodos hoy en día, como el método de Taylor, que pueden ser de precisión arbitraria si así se requiere. Actualmente la limitante es la capacidad de cómputo para hacer operaciones y el tamaño finito de la partición que representa a los números en una computadora.} de una condición inicial que represente el estado de un sistema en un momento dado. En esta tesis se propone extender el concepto de condición inicial al de vecindad inicial, lo cual se conoce como ``Transporte de Jets''. Es decir, el transporte encuentra el flujo de un conjunto de condiciones vecinas a una condición $\xo$ dada bajo un sistena de ecuaciones diferenciales ordinarias. Estas ideas ya se han abordado antes en los trabajos de Daniel Pérez-Palau \cite{Daniel2015, Perez2013, Perez2015} y en los trabajos de Makino y Berz \cite{Berz1991,Berz1998}, donde han explorado aplicaciones como la probabilidad de colisión entre satélites orbitando la Tierra, la navegación de naves espaciales via filtros de Kalman o  encontrar estructuras lagrangianas en el espacio fase. El nombre Transporte de Jets viene inspirado de la evolución de los jets en la física de partículas, que siempre vienen en paquete, tal como el jet de gluones descrito por Peskins en \cite{Peskin1996}.

La tesis está organizada de la siguiente manera: El capítulo \ref{chap:jt} presenta el concepto de transporte de jets como extensión de los métodos numéricos, específicamente montándolo sobre el método de Taylor, se construye un álgebra polinomial necesaria para las operaciones de las vecindades, y finalmente se pone a prueba al transporte en una serie de ejemplos hamiltonianos sencillos. El capítulo \ref{chap:jt_indicators} desarrolla una serie de indicadores dinámicos para el transporte de jets, donde se aprovecha el carácter paramétrico de las vecindades. El capítulo \ref{chap:CR3BP} estudia al problema circular de tres cuerpos, donde se analizan sus ecuaciones, sus puntos de equilibrio, sus constantes de movimiento y la estructura del espacio de configuraciones. Finalmente, el capítulo \ref{chap:results} presenta los resultados principales utilizando los conceptos del capítulo \ref{chap:jt} y los indicadores de \ref{chap:jt_indicators} en el problema circular de tres cuerpos desarrollado en el capítulo \ref{chap:CR3BP}. 


%Párrafo inicial con un poquito de historia, que de razones para alzar el estudio de ecuaciones diferenciles, específicamente los métodos numéricos. 
%Un poco la historia de cómo surge el jet transport y la motivación de querer usarlo. Aquí se menciona el objetivo 
%Hablar de la relación del JT con el P3C y por qué es un buen caso de estudio.