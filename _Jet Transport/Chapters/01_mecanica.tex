%%Relacionar mejor el inicio de este capítulo en relación a lo escrito en la introducción. Lo de ahorita es, en principio, preliminar. 

El objetivo de esta tesis es sacarle todo el provecho posible al transporte de jets y los indicadores que éste nos pueda regalar. Por esto, dos problemas físicos sencillos pero interesantes serán desarrollados en esta sección. El primero es el \textit{problema restringido de 3 cuerpos en una órbita circular (PR3COC)} y el segundo el \textit{sistema de Henón-Heiles (SHH)}. Antes de entrarle de lleno a estos problemas, creo que vale la pena recapitular un poco sobre la teoría en la que están basados; la mecánica analítica. Este capítulo se desarrollará como sigue: primero se hará un repaso de la mecánica lagrangiana y hamiltoniana muy al estilo de Landau y Lifshitz, específicamente en los capítulos I,II y VII de \cite{mechanics_landau_lifshitz}, luego, se desarrollará la teoría de PR3COC y, finalmente la de SHH.

\section{Mecánica}
\label{sec:mecanica}

Una buena forma de explorar cómo evoluciona un sistema dinámico que describa a un ente físico es encontrando las ecuaciones de movimiento que lo representa. Éste es el enfoque principal de la mecánica analítica de Hamilton y Lagrange\footnote{Claramente no son los únicos que desarrollaron esta teoría; grandes como Euler, Poisson o Liouville hicieron grandes aportaciones a la mecánica analítica, sin embargo, Hamilton y Lagrange cargan la bandera de la teoría gracias a que desarrollaron las ecuaciones de movimiento que llevan su nombre.} que, inspirados por Newton, encuentran la mejor descripción de sistemas macroscópicos basados en marcos de referencia inerciales\footnote{La elección de marcos de referencia inerciales dejan de lado la descripción relativista del mundo para esta teoría.}. Siempre que se hable de un sistema mecánico nos estaremos limitando a estas constricciones. 

Un sistema mecánico está descrito a partir de las partículas que se mueven en él y de la interacción entre ellas. Una \textit{partícula} es un cuerpo puntual\footnote{Aquí es donde el concepto de macroscópico toma sentido; un cuerpo es macroscópico si se puede ver como un conjunto de objetos puntuales o partículas. En la mecánica cuántica, por ejemplo, esto no es así.}, es decir, el tamaño y dimensiones de ésta son despreciados y sólamente importará para su descripción la \textit{cantidad de materia y/o carga} que ocupa, su \textit{posición} respecto a un marco de referencia y su \textit{velocidad} en un instante dado. Un cuerpo macroscópico, como un balón de fútbol, puede describirse como un conjunto de partículas que interaccionan entre sí y cómo interaccionan éstas con el mundo externo que es, a su vez, otro conjunto de partículas. Dichas interacciones definen la energía que pueden intercambiar estas partículas y generalmente se le conoce como el \textit{potencial}. El movimiento intrínseco de éstas también conlleva cierta energía que dependerá de la cantidad de materia y la velocidad de éstas; a dicha energía se le refiere usualmente como \textit{cinética}. 

Tomando como referencia un sistema de coordenadas cartesiano, podemos definir la posición de una partícula con su radio-vector $\mathbf{r}$ y su velocidad $\mathbf{v}$, que definimos como $\mathbf{v} := \frac{d \mathbf{r}}{dt} := \dot{\mathbf{r}}$. En general, una partícula vive en un espacio tridimensional así que necesita de tres coordenadas para describir su posición; un sistema de N partículas necesitará 3N coordenadas para describir la posición de todas. Resulta ser que si se conocen todas las posiciones y velocidades del sistema en un instante dado, se puede calcular, en principio, la dinámica de todo el sistema desde ese instante en adelante.\footnote{También se puede saber la dinámica desde ese instante hacia atrás; esa discusión se tendrá un poco más adelante.} 

%FIGURA!


Dada la naturaleza del sistema, a veces es posible describir un sistema de N partículas con menos de 3N coordenadas. Ejemplo de esto se ilustra en la figura \ref{fig:circular_scheme}, donde la posición de una partícula $m$ se puede determinar con precisión si se conoce el radio $R$ respecto al centro del círculo, en el cual $m$ se mueve en relación a $\theta$, el ángulo respecto a alguna línea de referencia que creamos conveniente. Para este ejemplo, basta únicamente de $\theta$ para describir la trayectoria de $m$ en lugar de las 3 coordenadas que se mencionaban en el párrafo anterior; podemos pensar que dicha partícula está \textit{restringida} a moverse sobre el círculo de radio $R$, es decir, ésta pierde la libertad de moverse arbitrariamente en cualquier punto del espacio tridimensional en donde vive. Con esto, definimos como \textit{grados de libertad} al número mínimo de coordenadas independientes con las cuales se puede definir la posición de todas las partículas de un sistema. 

\subsection{Principio de Mínima Acción}
\label{sec:least_action}

Un sistema tiene, en general, $s \leq N$ grados de libertad. Esto significa que se puede describir con $s$ cantidades independientes $\lbrace q_1, \ldots, q_s \rbrace$ que definen completamente la posición de todas las partículas. A estas cantidades se le llaman las \textit{coordenadas generalizadas} del sistema, y asociadas a ellas están las \textit{velocidades generalizadas} $\dot{q_i} = \frac{dq_i}{dt} \  \forall \ i \in [1,s]$. Parece ser que a la naturaleza\footnote{al menos a este nivel} le gusta seguir los caminos en los que la \textit{acción} sea mínima. Es decir, si a un instante $t_1$ el sistema ocupa las posiciones $q^{(1)} := (q_1(t_1), \ldots, q_s(t_1))$ y al instante $t_2$ las posiciones $q^{(2)}$,  debe existir una función $L$ con unidades de energía que dependa de las cantidades $q$, $\dot{q}$ y el tiempo tal que la integral 
\begin{equation}
 S = \int_{t_1}^{t_2} L(q,\dot{q},t) dt
 \label{eq:action}
\end{equation}
sea mínima.\footnote{Notemos que $S$ tiene unidades de acción: $[ S ] = [energía \times tiempo]$.} Esta condición se conoce como el \textit{principio de mínima acción} o de \textit{principio de Hamilton}. A $L$ se le conoce como el \textit{lagrangiano} del sistema.

Supongamos que $q = q(t)$ es la trayectoria que minimiza a \ref{eq:action}, entonces cualquier variación $\delta q(t)$ a $q(t)$ hace que $S$ incremente. Tomaremos, sin pérdida de generalidad, el caso $s = 1$. Obviamente, $\delta q(t_1) = \delta q(t_2) = 0$ ya que, aunque queremos que la trayectoria tome un camino alternativo, éste debe empezar y terminar en los mismos puntos que $q(t)$. 

Con esto, la variación en $S$ será 
\begin{align*}
 \delta S = \int_{t_1}^{t_2} \delta L dt = \int_{t_1}^{t_2} \left( \pder{L}{q} \delta q + \pder{L}{ \dot{q} } \delta \dot{q} + \cancelto{0}{\mathcal{O}(\delta q^2)} \right) dt
\end{align*}

donde buscamos $\delta S = 0$ para que $S$ sea mínimo. Observando que $\delta \dot{q} = \delta \frac{dq}{dt}$ e integrando por partes, tenemos que 

\begin{align*}
 \delta S &= \int_{t_1}^{t_2} \pder{L}{q} \delta q dt + \pder{L}{\dot{q}} \cancelto{0}{\delta q} \rvert_{t_1}^{t_2} - \int_{t_1}^{t_2} \frac{d}{dt} \left( \pder{L}{\dot{q}} \right) \delta q dt\\ 
 &= \int_{t_1}^{t_2} \left( \pder{L}{q} - \frac{d}{dt} \left( \pder{L}{ \dot{q} } \right) \delta q dt \right) = 0.
\end{align*}

Así, imponiendo la condición de que el integrando sea idénticamente cero, obtenemos las \textit{ecuaciones de movimiento de Lagrange}
\begin{equation}
 \pder{L}{d q_i} - \frac{d}{dt} \left( \pder{L}{ \dot{q_i} } \right) = 0 \ \forall \ i \in [1, \ldots, s],
 \label{eq:lagrange_equations}
\end{equation}

que son $s$ ecuaciones diferenciales ordinarias de segundo orden y, por tanto, $2s$ constantes o condiciones iniciales $\{ q_{1_0}, \ldots, q_{s_0}, \dot{q_{1_0}}, \ldots, \dot{q_{s_0}} \}$\footnote{Se entiende que $q_{i_0} = q_i(t_0)$.} son necesarias para resolverlas. Con esto, la pregunta primordial es: ¿Qué forma tiene $L$?

\subsection{Lagrangiano de la partícula libre}

Tomemos como primer caso a la partícula libre (de interacciones). Ésta no tiene constricciones ni interactúa con nadie más y, por tanto, tiene 3 grados de libertad. Tomaremos como marco de referencia inercial las coordenadas rectangulares definidas por $\{ \hat{\imath}, \hat{\jmath}, \hat{k} \}$ al cual llamaremos $\mathcal{K}$. Es importante para la descripción notar que en un marco de referencia inercial el espacio es homogéneo e isotrópico y el tiempo homogéneo. 

Así, por la homogeneidad del éstos, $L$ no puede depender explícitamente de $\mathbf{r}$ ni de $t$ y, por tanto, $L(\mathbf{r},\mathbf{v},t) = L(\mathbf{v})$. Además, por lo isotrópico del espacio, debería ser indistinto para $L$ la dirección en la que la partícula se esté moviendo, de modo que la dependencia deberá sobre la magnitud de la velocidad, i.e., $L = L(\mathbf{v}) = L(v^2)$.

Por (\ref{eq:lagrange_equations}), tenemos que 
 \begin{align*}
 \pder{L(\mathbf{v}) }{ \mathbf{r} } = 0 &\implies \frac{d}{dt}\pder{L(\mathbf{v})}{\mathbf{v}} = \frac{d}{dt} \frac{dL(\mathbf{v})}{d\mathbf{v}} = 0 \\ 
 &\therefore \mathbf{v} = \text{constante}.
 \end{align*}


Hay que resaltar dos detalles sobre el lagrangiano para continuar con esta descripción:
\begin{enumerate}

  \item Dos lagrangianos $L$ y $\mathcal{L}$ que difieren bajo una función $f(q,t)$ de la forma $\mathcal{L}(q,\dot{q},t) = L(q,\dot{q},t) + \frac{d f(q,t)}{dt}$ son equivalentes ya que $\delta S_{L} = \delta S_{\mathcal{L}} = 0$\footnote{$ \delta S_{\mathcal{L}} = \cancelto{0}{\delta S_L} + \delta f(q^{(2)},t_2) - \delta f(q^{(1)},t_1) = \pder{f(q^{(2)},t_2)}{q} \cancelto{0}{ \delta q^{(2)} } - \pder{f(q^{(1)},t_1)}{q} \cancelto{0}{ \delta q^{(1)} } = 0$.} y, por tanto, la forma de las ecuaciones definidas por (\ref{eq:lagrange_equations}) no cambia.

 \item Si dos lagrangianos $L_1$ y $L_2$ no interactúan entre sí en un mismo sistema, entonces ambos se pueden englobar en uno mismo como la suma indivudual de sus partes, i.e.
 \begin{equation}
  L = L_1 + L_2 .
  \label{eq:lagrangian_addititivy}
 \end{equation}
 
\end{enumerate}

Por lo primero, podemos encontrar la forma explícita de $L$ para la partícula libre. Sabemos que en el marco $\mathcal{K}$ la partícula se mueve con velocidad $\mathbf{v}$. Si otro marco inercial $\mathcal{K}'$ se mueve a $\mathbf{\epsilon} \ll \mathbf{v}$ respecto a $\mathcal{K}$, entonces la partícula se con velocidad $\mathbf{v}' = \mathbf{v} + \mathbf{\epsilon}$ en $\mathcal{K}'$. Así,
\begin{align*}
 \mathcal{L} = L(v'^2) = L(v^2 + 2 \mathbf{v} \cdot \mathbf{\epsilon} + \epsilon^2) \\
 \implies  L(v'^2) = L(v^2) + \pder{L}{v^2} 2 \mathbf{v} \cdot \mathbf{\epsilon} + \cancelto{0}{\mathcal{O}(\epsilon^2)}.
\end{align*}
Como $\mathcal{L}$ y $L$ sólo pueden diferir por la derivada temporal de alguna función $f(\mathbf{v},t)$, entonces
\begin{align*}
 \pder{L}{v^2} = \alpha
\end{align*}
y, por tanto
\begin{equation}
 L(v^2) = \alpha v^2 = \frac{1}{2}m v^2,
\end{equation}
con $m$ la masa de la partícula, consiguiendo que el lagrangiano tenga unidades de energía.

Por lo segundo, como las partículas libres, por definición, no interactúan entre sí, el lagrangiano de un sistema de $N$ partículas libres será
\begin{equation}
 L(\mathbf{v}_1, \ldots, \mathbf{v}_N) = L_1(\mathbf{v}_1) + \ldots + L_N(\mathbf{v}_N) = \sum_{i=1}^N \frac{1}{2}m_i v_i^2.
 \label{eq:kinetic}
\end{equation}

Notemos que este particular lagrangiano describe la energía del movimiento de las partículas cuando no interaccionan entre ellas; por esto, en el panorama más general, se le conoce como la \textit{energía cinética} del sistema: $L_{libre}(\mathbf{v}) := T(\mathbf{v})$.

\subsection{Lagrangiano de un sistema cerrado} 

Un sistema de N partículas se llama cerrado si éstas interaccionan únicamente entre ellas; es decir, no hay agentes externos que alteren al sistema. Vimos que la energía cinética está dada por (\ref{eq:kinetic}) y, así, al lagrangiano se le debe agregar la interacción entre partículas o \textit{energía potencial}. Resulta que si el sistema es cerrado, el potencial será una función que depende únicamente de la posición que cada partícula tiene en un instante dado. Nombraremos $-U$ a esta función, y así
\begin{equation}
 L(\mathbf{r},\mathbf{v}) = \sum_{i=1}^N \frac{1}{2}m_i v_i^2 -U(\mathbf{r_1},\ldots,\mathbf{r_N}).
 \label{eq:lagrangian_cartesian_closed}
\end{equation} 

Hay tres puntos que creo vale la pena mencionar. Primero, Como $U$ es función de las posiciones únicamente entonces, si la posición de cualquier partícula cambia, el potencial  afecta instantaneamente a todas las demás. Segundo, si suponiéramos que dichas interacciones no son instantáneas, entonces éstas se propagarían con cierta velocidad y, por tanto, la invarianza de las ecuaciones sobre marcos de referencia inerciales ya no sería válida. Tercero, sustituyendo (\ref{eq:lagrangian_cartesian_closed}) en (\ref{eq:lagrange_equations}) tenemos
\begin{align}
 -\pder{U(\mathbf{r_1}, \ldots, \mathbf{r_N})}{\mathbf{r}_i} &= \frac{d}{dt} \pder{}{\mathbf{v}_i} \sum_{i=1}^N \frac{1}{2}m_i v_i^2 \nonumber \\
 &\therefore m_i \ddot{\mathbf{r}_i} = -\pder{U(\mathbf{r_1}, \ldots, \mathbf{r_N})}{\mathbf{r}_i} =: \mathbf{F}_i(\mathbf{r}) \nonumber \\
 &\therefore \mathbf{F}(\mathbf{r}) = \sum \mathbf{F}_i(\mathbf{r}) = - \nabla U(\mathbf{r}),
 \label{eq:newton_second_law}
\end{align}
que corresponde a la \textit{segunda ley de Newton} para sistemas cerrados; de ahí la elección del ``$-$'' para la función $U$.

Ahora, el lagrangiano dado por (\ref{eq:lagrangian_cartesian_closed}) fue trabajado en coordenadas rectangulares dado que la partícula libre no tiene constricciones del espacio. Como el potencial puede que restringa las trayectorias en el que las partículas se mueven, es posible que el sistema pueda ser descrito en $s < 3N$  coordenadas generalizadas en donde existe una función $\mathbf{g}_j: \mathbb{R}^{s} \to \mathbb{R}^{N}, \ j \in \{1,2,3\}$ tal que $g_{j,i}(q) = x_{j,i}$, donde $\mathbf{r}_i = x_{1,i} \hat{\imath} + x_{2,i} \hat{\jmath} + x_{3,i} \hat{k}$, $\dot{x}_{j,i} = \sum_k \pder{\mathbf{g}_j(q)}{q_k} \dot{q_k}$ y $q \in \mathbb{R}^s$. Metiendo todo esto en el lagrangiano obtenemos
\begin{equation}
 L(q,\dot{q}) = \frac{1}{2}\sum_{i,k} a_{i,k}(q) \dot{q_i}\dot{q_k} - U(q),
 \label{eq:lagrangian_closed_generalized}
\end{equation}

el lagrangiano en coordenadas generalizadas, donde $\mathbf{a}(q)$ es la función que sale de sustituir la transformación en $T(\dot{\mathbf{r}}) = \frac{1}{2}\sum m_i \left( \dot{x}_{1,i}^2 + \dot{x}_{2,i}^2 + \dot{x}_{3,i}^2 \right)$. Es interesante destacar que en coordenadas generalizadas la energía cinética puede depender de las posiciones generalizadas y es, como antes, cuadrática en las velocidades. 

\subsection{Lagrangiano de un sistema abierto}

Un sistema abierto es aquel que puede interactuar con sus alrededores. Si un sistema $A$ interactúa con un sistema $B$ decimos, desde la perspectiva de $A$, que éste se mueve en un campo externo dado por $B$. Suponiendo que el sistema conformado por $A$ y $B$ juntos es cerrado, entonces tendrá un lagrangiano 
\begin{equation}
 L(q_A,q_B,\dot{q}_A,\dot{q}_B) = T_A(q_A,\dot{q}_A) + T_B(q_B,\dot{q}_B) - U(q_A,q_B),
\end{equation}
con $U$ un potencial que considera la interacción de todas las partículas $q_A$ y $q_B$.

Como no nos interesa la treyectoria de las partículas de $B$ sino únicamente en cómo afecta a $A$, podemos sustituir  en el lagrangiano $q_B \to q_b(t)$, la solución explicita en $t$ para $B$, obteniendo
\begin{align*}
 L(q_A,\dot{q}_A,t) = T_A(q_A,\dot{q_A}) + \cancelto{\frac{df}{dt}}{T_B(t)} - U(q_A,q_B(t)).
\end{align*}

Así, un sistema que interactúa con un campo externo difiere con uno cerrado sólo en el potencial, que tendrá una dependencia explícita del tiempo
\begin{equation}
 L(q,\dot{q},t) = T(q,\dot{q}) - U(q,t).
 \label{eq:lagrangian_open_generalized}
\end{equation}

\subsection{Marcos de referencia no inerciales}
\label{sec:ficticious_forces}

Se ha hablado hasta aquí de sistemas descritos por marcos de referencia inerciales. Dicha necesidad emerge de la invarianza ante estos y de la invarianza, así, de las ecuaciones de movimiento. Sin embargo, hay sistemas donde resulta más cómoda la descripción si el marco de referencia tiene ciérta aceleración respecto a los marcos de referencia inerciales con los que hemos trabajado. Un auto que busca tomar una curva en una carretera desde el punto de vista del piloto o un satélite que navega las órbitas entre dos planetas desde el punto del satélite son dos buenos ejemplos de esto. La aceleración dada por el marco de referencia brindará, según la segunda ley de Newton ($\ddot{\mathbf{r}}_{ref} \neq 0$), una fuerza extra al sistema siempre que un móvil masivo interactúe con él. A este tipo de fuerzas se le llaman \textit{ficticias}, ya que no son provocadas por campos externos ni interacción entre partículas, únicamente por la elección de un sistema de referencia que no va a velocidad constante. 

Veremos como ejemplo las fuerzas ficticias que aparecen en un marco de referencia en rotación respecto a uno inercial. Ésto ilustará cómo trabajar en marcos de referencia no inerciales y se usará posteriormente para el problema de los tres cuerpos. 

\subsubsection{Marcos rotantes}

Sea un marco que rota respecto algún eje de simetría con velocidad angular $\theta(t)$. Si basamos un marco de rerefencia inercial tal que el eje de rotación coincida con el eje $z$, entonces como se muestra en la figura \ref{fig:}, la relación entre ambos se puede expresar en coordenadas cilíndricas como
\begin{align}
 \hat{\imath}_r(t) &= \cos \theta(t) \hat{\imath}_i + \sin \theta(t) \hat{\jmath}_i \nonumber \\
 \hat{\jmath}_r(t) &= -\sin \theta(t) \hat{\imath}_i + \cos \theta(t) \hat{\jmath}_i \\
 \hat{k}_r(t) &= \hat{k}_i \nonumber
 \label{eq:rotating_unitaries}
\end{align}
donde los subíndices ``$r$'' y ``$i$'' se refieren a los marcos de referencia rotado e inercial, respectivamente. Así, el cambio de los unitarios respecto al tiempo son
\begin{align}
 \frac{d}{dt} \hat{\imath}_r(t) &= \dot{\theta}(t) \left( -\sin \theta(t) \hat{\imath}_i + \cos \theta(t) \hat{\jmath}_i  \right) \nonumber \\
 \frac{d}{dt} \hat{\jmath}_r(t) &= \dot{\theta}(t) \left( -\cos \theta(t) \hat{\imath}_i - \sin \theta(t) \hat{\jmath}_i \right) \\
 \frac{d}{dt} \hat{k}_r &= 0 \nonumber 
 \label{eq:rotating_unitaries_derivs}
\end{align}
Siguiendo la construcción de \cite{wiki_rotating_frame}, definimos al vector de rotación $\mathbf{\Omega} := \left( 0, 0, \dot{\theta} \right)$ y, con éste, cualquier término de (\ref{eq:rotating_unitaries_derivs}) se puede expresar como 
\begin{equation}
 \frac{d}{dt}\hat{u} = \mathbf{\Omega} \times \hat{u}.
\end{equation}

Sea entonces $\mathbf{f}(t) = f_x(t) \hat{\imath} + f_y(t) \hat{j} + f_z(t) \hat{k}$ una cantidad definida en el marco que rota, la descripción de su derivada desde el marco de referencia inercial es, por la regla del producto,
\begin{equation*}
 \left( \frac{d \mathbf{f}}{dt} \right)_i = \left( \frac{d\mathbf{f}}{dt} \right)_r  + \mathbf{\Omega} \times \mathbf{f}
\end{equation*}
y, por tanto,
\begin{equation}
 \left(\frac{d}{dt}\right)_i := \left( \frac{d}{dt} \right)_r + \mathbf{\Omega} \times
 \label{eq:rotating_derivative}
\end{equation}
es un operador que expresa la derivada de $\mathbf{f}$ en el marco de referencia inercial.

Con (\ref{eq:rotating_derivative}) se pueden expresar la velocidad 
\begin{equation}
 \mathbf{v}_i = \dot{\mathbf{r}}_i = \mathbf{v}_r + \mathbf{\Omega} \times \mathbf{r}_r,
 \label{eq:rotating_velocity}
\end{equation}
y la aceleración
\begin{align}
 \mathbf{a}_i = \ddot{\mathbf{r}}_i &= \left[ \left( \frac{d}{dt}\right)_r + \mathbf{\Omega} \times \right]\left[ \mathbf{v}_r + \mathbf{\Omega} \times \mathbf{r}_r \right] \nonumber \\
 \nonumber \\
 \therefore \ddot{\mathbf{r}}_i &= \ddot{\mathbf{r}}_r + 2\mathbf{\Omega} \times \mathbf{\dot{r}}_r + \mathbf{\Omega} \times \left( \mathbf{\Omega} \times \mathbf{r}_r \right) + \dot{\mathbf{\Omega}} \times \mathbf{r}_r
 \label{eq:rotating_acceleration}
\end{align}
en relación al marco de referencia inercial del sistema. 

Con esto, podemos ver que si una partícula de masa $m$ tiene una aceleración $\ddot{\mathbf{r}}$ en el marco en rotación, ésta sentirá una serie de fuerzas ficticias si es vista desde un marco inercial. Al término ``$2 m \mathbf{\Omega} \times \dot{\mathbf{r}}$'' se le conoce como \textit{fuerza centrípeta}, a ``$m \mathbf{\Omega} \times ( \mathbf{\Omega} \times \mathbf{r} )$'' como la \textit{fuerza de Coriolis} y a ``$m \dot{\mathbf{\Omega}} \times \mathbf{r}$'' como \textit{fuerza de Euler}.

Notemos que si la rotación es uniforme entonces $\dot{\mathbf{\Omega}} = 0$. Además, si el sistema es cerrado en el marco rotativo, entonces, por (\ref{eq:newton_second_law}), (\ref{eq:rotating_acceleration}) se reduce a 
\begin{equation}
 \ddot{\mathbf{r}}_i - 2\mathbf{\Omega} \times \dot{\mathbf{r}}_r = \nabla \left( \frac{1}{m}U(\mathbf{r}_r) +  \mathbf{\Omega}^2 \mathbf{r}_r^2 \right).
\end{equation}

Salvo por la fuerza de Coriolis, que vuelve a éste un sistema abierto, se tiene que las ecuaciones de movimiento expresan un sistema cerrado con el nuevo potencial $U_{tot}(\mathbf{r}) = U(\mathbf{r}_r) + m \mathbf{\Omega}^2 \mathbf{r}_r^2$.

%Revisar ésto bien en sintaxis y signos y ver qué nos interesa en cada caso entre "i" y "r". 
%Meter imágenes.

\subsection{Ecuaciones de Hámilton}
\label{sec:hamilton}

Estableciendo el principio de mínima acción encontramos qué condiciones debe satisfacer el lagrangiano. Ésta condición define las ecuaciones de movimiento del sistema en cuestión en términos de las posiciones, las velocidades y el tiempo $q, \dot{q}$ y $t$, respectivamente. Sin embargo, nada dice que éstas sean las únicas cantidades para describir a dicho sistema. En la formulación de las ecuaciones de movimiento de Hámilton, un sistema es descrito por $q,p$ y $t$, las posiciones generalizadas, el momento generalizado y el tiempo, respectivamente. De manera similar a lagrangiano, que es una función $L = L(q,\dot{q},t)$, llamaremos \textit{hamiltoniano} a la función que represente las ecuaciones de movimiento equivalentes, donde $H = H(q,p,t)$. Resulta conveniente muchas veces trabajar en este esquema ya que el hanmiltoniano representa a la energía total mecánica de este. Se puede pasar de un esquema al otro vía una transformada de Legandre.

Sabemos que 
\begin{equation}
 dL = \sum_i \pder{L}{q_i} d q_i + \sum_i \pder{L}{\dot{q_i} }  d\dot{q_i}
 \label{eq:lagrangian_differential}
\end{equation}
y, definiendo 
\begin{equation}
 p_i := \pder{L}{\dot{q}_i}
\end{equation} 
obtenemos, sustituyendo en (\ref{eq:lagrange_equations}) que
\begin{equation}
 \pder{L}{q_i} = \dot{p}_i.
 \label{eq:dot_momentum}
\end{equation}

Por otro lado, por la regla del producto, observamos que $d\left( \sum p_i \dot{q}_i \right) = \sum p_i d \dot{q}_i + \sum d p_i \dot{q}_i$. Usando esto, y mezclándolo con (\ref{eq:dot_momentum}) y (\ref{eq:lagrangian_differential}), obtenemos
\begin{align}
 dL &= \sum_i \dot{p}_i d q_i + \sum_i p_i d \dot{q}_i \nonumber \\
 \implies d\left(\sum_i p_i \dot{q}_i - L \right) &= - \sum_i \dot{p}_i d q_i + \sum_i \dot{q}_i d p_i.
 \label{eq:legandre_ham_lag}
\end{align}

Como $H = H(q,p,t)$, entonces 
\begin{equation}
 dH = \sum_i \pder{H}{q_i} dq_i + \sum_i \pder{H}{dp_i} dp_i,
 \label{eq:hamilton_differential}
\end{equation}
y, comparándolo con (\ref{eq:legandre_ham_leg}), concluimos que el hamiltoniano está dado por
\begin{equation}
 H = \sum_i p_i \dot{q}_i - L
 \label{eq:hamiltonian_legandre}
\end{equation}
y define las ecuaciones de movimiento
\begin{align}
 &\pder{H}{q_i} = - \dot{p}_i && \pder{H}{p_i} = \dot{q}_i.
\end{align}

Como el hamiltoniano representa la energía total del sistema, nos puede interesar cómo cambia en una trayectoria dada; así 
\begin{equation*}
 \frac{dH}{dt} = \pder{H}{t} + \sum_i \pder{H}{p_i}\dot{p}_i + \sum_i \pder{H}{q_i}\dot{q}_i = \pder{H}{t} + \sum_i - \cancel{\dot{p}_i\dot{q}_i} + \cancel{\dot{p}_i\dot{q}_i} = \pder{H}{t}.
\end{equation*}
Observamos que si $H$ no depende explícitamente del tiempo entonces es constante, i.e.,
\begin{equation*}
 H(p,q) = \text{constante} \ \forall (q,p) \in q(t).
\end{equation*} Esto no es más que la conservación de la energía mecánica para un sistema cerrado.\footnote{Equivalentemente al lagrangiano, si el hamiltoniano no depende explícitamente del tiempo será un sistema cerrado; si sí, entonces será abierto.}

Otro punto interesante sobre la descripción hamiltoniana es que $H$ puede expresarse de una manera más elegante notando que la energía cinética es cuadrática en las velocidades, entonces
\begin{align*}
 \sum_i \dot{q}_i \pder{L}{\dot{q_i}} = \sum_i \dot{q}_i \pder{T}{\dot{q_i}} = 2 T.
\end{align*}
Sustituyendo esto último en (\ref{eq:hamiltonian_legandre}) obtenemos
\begin{align}
 H = 2T - L = 2T - \left( T - U \right) \nonumber \\ 
 \nonumber \\
 \therefore H(q,p,t) = T(q,p) + U(q,t).
 \label{eq:hamiltonian}
\end{align}
El hamiltoniano de un sistema es la suma de las energías cinética y potencial de éste. 

\section{Problema restringido de 3 cuerpos en una órbita circular (PR3COC)}

El problema de tres cuerpos interactuando en campos gravitacionales ha sido de gran interés desde hace más de 300 años. Newton, en el siglo $XVII$, publica en su Philosophiæ Naturalis Principia Mathematica \cite{Principia} los primeros intentos ``serios'' para describir el movimiento entre el sol, la luna y la Tierra. Resulta que principios del siglo $XX$, el matemático Karl Sundman resolvió el problema con series de potencias convergentes, aunque no encontró ninguna función explícita que la represente. Sin embargo, el gran matemático Henrí Poincaré vio, en el siglo $XIV$, algo que no se había definido antes: \textit{el caos}. El problema de tres cuerpos tiene comportamiento caótico y es justo esa la motivación a desarrollarlo ya que las herramientas que se desarrollarán en el capítulo \ref{chapter:jet_transport} se aprovecharán de su naturaleza caótica. 

Plantear el problema resulta trivial ya que el potencial de interacción se da entre pares de cuerpos, así, siguiendo el potencial gravitacional de Newton, 
\begin{align}
 U_{M_1}(\mathbf{r}_1,\mathbf{r}_2,\mathbf{r}_3) &= -G \frac{M_1 M_2}{r_{1,2}} - G \frac{M_1 M_3}{r_{1,3}} \\
 U_{M_2}(\mathbf{r}_1,\mathbf{r}_2,\mathbf{r}_3) &= -G \frac{M_2 M_1}{r_{2,1}} - G \frac{M_2 M_3}{r_{2,3}} \\
 U_{M_3}(\mathbf{r}_1,\mathbf{r}_2,\mathbf{r}_3) &= -G \frac{M_3 M_1}{r_{3,1}} - G \frac{M_3 M_2}{r_{3,2}}
 \label{eq:3body_potential}
\end{align}

con $G$ la constante de gravitación universal, $\mathbf{r}_i = x_i \hat{\imath} + y_i \hat{\jmath} + z_i \hat{k}$ la posición de la i-ésima partícula, $M_i$ su masa y $r_{i,j} = \norm{ \mathbf{r}_i - \mathbf{r}_j}$ la distancia entre las partículas $M_i$ y $M_j$\footnote{Notemos la simetría $r_{i,j} = r_{j,i}$, aunque $\mathbf{r}_{i,j} = \mathbf{r}_i - \mathbf{r}_j = \mathbf{r}_j - \mathbf{r}_i  = - \mathbf{r}_{j,i}$.}. 

La energía cinética del problema la definimos de la manera usual 
\begin{equation}
 T(\dot{\mathbf{r}}_1,\dot{\mathbf{r}}_2,\dot{\mathbf{r}}_3) = \sum_{i=1}^3 M_i \dot{r}_i^2,
 \label{eq:3body_cinetic}
\end{equation}
lo cual define un sistema de 6 ecuaciones de movimiento dadas por (\ref{eq:newton_second_law})
\begin{equation}
 M_i \ddot{\mathbf{r}_i} = - G M_i \sum_{j\neq i} \frac{M_j}{r_{i,j}^2} \hat{\mathbf{r}}_{i,j},
 \label{eq:3body_eqs_motion}
\end{equation}
con $\hat{\mathbf{r}}_i = \frac{\mathbf{r}_i}{r_i}$ un vector unitario en la dirección de $\mathbf{r}_i$.

Éste es el caso más general del problema de tres cuerpos ya que ninguna restricción ha sido impuesta hasta el momento y, por tanto, no podemos hablar aún del problema restringido a órbitas circulares. Para ésto, vale la pena resolver primero el caso de dos cuerpos y luego regresar al de tres y ver a qué condiciones estará sujeto. 

