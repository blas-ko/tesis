%%Relacionar mejor el inicio de este capítulo en relación a lo escrito en la introducción. Lo de ahorita es, en principio, preliminar. 

El objetivo de esta tesis es sacarle todo el provecho posible al transporte de jets y los indicadores que éste nos pueda regalar. Por esto, dos problemas físicos sencillos pero interesantes serán desarrollados en esta sección. El primero es el \textit{problema restringido de tres cuerpos en órbitas circulares (PR3COC)} y el segundo el \textit{sistema de Henón-Heiles (SHH)}. Antes de entrarle de lleno a estos problemas, creo que vale la pena recapitular un poco sobre la teoría en la que están basados; la mecánica analítica. Este capítulo se desarrollará como sigue: primero se hará un repaso de la mecánica lagrangiana y hamiltoniana muy al estilo de Landau y Lifshitz, específicamente en los capítulos I,II y VII de \cite{mechanics_landau_lifshitz}, luego, se desarrollará la teoría de PR3COC y, finalmente la de SHH.


\subsection{Mecánica}
\label{sec:mecanica}

Una buena forma de explorar cómo evoluciona un sistema dinámico que describa a un ente físico es encontrando las ecuaciones de movimiento que lo representa. Éste es el enfoque principal de la mecánica analítica de Hamilton y Lagrange\footnote{Claramente no son los únicos que desarrollaron esta teoría; grandes como Euler, Poisson o Liouville hicieron grandes aportaciones a la mecánica analítica, sin embargo, Hamilton y Lagrange cargan la bandera de la teoría gracias a que desarrollaron las ecuaciones de movimiento que llevan su nombre.} que, inspirados por Newton, encuentran la mejor descripción de sistemas macroscópicos basados en marcos de referencia inerciales\footnote{La elección de marcos de referencia inerciales dejan de lado la descripción relativista del mundo para esta teoría.}. Siempre que se hable de un sistema mecánico nos estaremos limitando a estas constricciones. 

Un sistema mecánico está descrito a partir de las partículas que se mueven en él y de la interacción entre ellas. Una \textit{partícula} es un cuerpo puntual\footnote{Aquí es donde el concepto de macroscópico toma sentido; un cuerpo es macroscópico si se puede ver como un conjunto de objetos puntuales o partículas. En la mecánica cuántica, por ejemplo, esto no es así.}, es decir, el tamaño y dimensiones de ésta son despreciados y sólamente importará para su descripción la \textit{cantidad de materia y/o carga} que ocupa, su \textit{posición} respecto a un marco de referencia y su \textit{velocidad} en un instante dado. Un cuerpo macroscópico, como un balón de fútbol, puede describirse como un conjunto de partículas que interaccionan entre sí y cómo interaccionan éstas con el mundo externo que es, a su vez, otro conjunto de partículas. Dichas interacciones definen la energía que pueden intercambiar estas partículas y generalmente se le conoce como el \textit{potencial}. El movimiento intrínseco de éstas también conlleva cierta energía que dependerá de la cantidad de materia y la velocidad de éstas; a dicha energía se le refiere usualmente como \textit{cinética}. 

Tomando como referencia un sistema de coordenadas cartesiano, podemos definir la posición de una partícula con su radio-vector $\mathbf{r}$ y su velocidad $\mathbf{v}$, que definimos como $\mathbf{v} := \frac{d \mathbf{r}}{dt} := \dot{\mathbf{r}}$. En general, una partícula vive en un espacio tridimensional así que necesita de tres coordenadas para describir su posición; un sistema de N partículas necesitará 3N coordenadas para describir la posición de todas. Resulta ser que si se conocen todas las posiciones y velocidades del sistema en un instante dado, se puede calcular, en principio, la dinámica de todo el sistema desde ese instante en adelante.\footnote{También se puede saber la dinámica desde ese instante hacia atrás; esa discusión se tendrá un poco más adelante.} 

%FIGURA!


Dada la naturaleza del sistema, a veces es posible describir un sistema de N partículas con menos de 3N coordenadas. Ejemplo de esto se ilustra en la figura \ref{fig:circular_scheme}, donde la posición de una partícula $m$ se puede determinar con precisión si se conoce el radio $R$ respecto al centro del círculo, en el cual $m$ se mueve en relación a $\theta$, el ángulo respecto a alguna línea de referencia que creamos conveniente. Para este ejemplo, basta únicamente de $\theta$ para describir la trayectoria de $m$ en lugar de las 3 coordenadas que se mencionaban en el párrafo anterior; podemos pensar que dicha partícula está \textit{restringida} a moverse sobre el círculo de radio $R$, es decir, ésta pierde la libertad de moverse arbitrariamente en cualquier punto del espacio tridimensional en donde vive. Con esto, definimos como \textit{grados de libertad} al número mínimo de coordenadas independientes con las cuales se puede definir la posición de todas las partículas de un sistema. 

\subsubsection{Principio de Mínima Acción}
\label{sec:least_action}

Un sistema tiene, en general, $s \leq N$ grados de libertad. Esto significa que se puede describir con $s$ cantidades independientes $\lbrace q_1, \ldots, q_s \rbrace$ que definen completamente la posición de todas las partículas. A estas cantidades se le llaman las \textit{coordenadas generalizadas} del sistema, y asociadas a ellas están las \textit{velocidades generalizadas} $\dot{q_i} = \frac{dq_i}{dt} \  \forall \ i \in [1,s]$. Parece ser que a la naturaleza\footnote{al menos a este nivel} le gusta seguir los caminos en los que la \textit{acción} sea mínima. Es decir, si a un instante $t_1$ el sistema ocupa las posiciones $q^{(1)} := (q_1(t_1), \ldots, q_s(t_1))$ y al instante $t_2$ las posiciones $q^{(2)}$,  debe existir una función $L$ con unidades de energía que dependa de las cantidades $q$, $\dot{q}$ y el tiempo tal que la integral 
\begin{equation}
 S = \int_{t_1}^{t_2} L(q,\dot{q},t) dt
 \label{eq:action}
\end{equation}
sea mínima.\footnote{Notemos que $S$ tiene unidades de acción: $[ S ] = [energía \times tiempo]$.} Esta condición se conoce como el \textit{principio de mínima acción} o de \textit{principio de Hamilton}. A $L$ se le conoce como el \textit{lagrangiano} del sistema.

Supongamos que $q = q(t)$ es la trayectoria que minimiza a \ref{eq:action}, entonces cualquier variación $\delta q(t)$ a $q(t)$ hace que $S$ incremente. Tomaremos, sin pérdida de generalidad, el caso $s = 1$. Obviamente, $\delta q(t_1) = \delta q(t_2) = 0$ ya que, aunque queremos que la trayectoria tome un camino alternativo, éste debe empezar y terminar en los mismos puntos que $q(t)$. 

Con esto, la variación en $S$ será 
\begin{align*}
 \delta S = \int_{t_1}^{t_2} \delta L dt = \int_{t_1}^{t_2} \left( \pder{L}{q} \delta q + \pder{L}{ \dot{q} } \delta \dot{q} + \cancelto{0}{\mathcal{O}(\delta q^2)} \right) dt
\end{align*}

donde buscamos $\delta S = 0$ para que $S$ sea mínimo. Observando que $\delta \dot{q} = \delta \frac{dq}{dt}$ e integrando por partes, tenemos que 

\begin{align*}
 \delta S &= \int_{t_1}^{t_2} \pder{L}{q} \delta q dt + \pder{L}{\dot{q}} \cancelto{0}{\delta q} \rvert_{t_1}^{t_2} - \int_{t_1}^{t_2} \frac{d}{dt} \left( \pder{L}{\dot{q}} \right) \delta q dt\\ 
 &= \int_{t_1}^{t_2} \left( \pder{L}{q} - \frac{d}{dt} \left( \pder{L}{ \dot{q} } \right) \delta q dt \right) = 0.
\end{align*}

Así, imponiendo la condición de que el integrando sea idénticamente cero, obtenemos las \textit{ecuaciones de movimiento de Lagrange}
\begin{equation}
 \pder{L}{d q_i} - \frac{d}{dt} \left( \pder{L}{ \dot{q_i} } \right) = 0 \ \forall \ i \in [1, \ldots, s],
 \label{eq:lagrange_equations}
\end{equation}

que son $s$ EDO de segundo orden y, por tanto, $2s$ constantes o condiciones iniciales $\{ q_{1_0}, \ldots, q_{s_0}, \dot{q_{1_0}}, \ldots, \dot{q_{s_0}} \}$\footnote{Se entiende que $q_{i_0} = q_i(t_0)$.} son necesarias para resolverlas. Con esto, la pregunta primordial es: ¿Qué forma tiene $L$?

Tomemos como primer caso a la partícula libre. Ésta no tiene constricciones ni interactúa con nadie más y, por tanto, tiene 3 grados de libertad. Tomaremos como marco de referencia inercial las coordenadas rectangulares definidas por $\{ \hat{i}, \hat{j}, \hat{k} \}$ al cual llamaremos $\mathcal{K}$. Es importante para la descripción notar que en un marco de referencia inercial el espacio es homogéneo e isotrópico y el tiempo homogéneo. 

Así, por la homogeneidad del éstos, $L$ no puede depender explícitamente de $\mathbf{r}$ ni de $t$ y, por tanto, $L(\mathbf{r},\mathbf{v},t) = L(\mathbf{v})$. Además, por lo isotrópico del espacio, debería ser indistinto para $L$ la dirección en la que la partícula se esté moviendo, de modo que la dependencia deberá sobre la magnitud de la velocidad, i.e., $L = L(\mathbf{v}) = L(v^2)$.

Por (\ref{eq:lagrange_equations}), tenemos que 
 \begin{align*}
 \pder{L(\mathbf{v}) }{ \mathbf{r} } = 0 &\implies \frac{d}{dt}\pder{L(\mathbf{v})}{\mathbf{v}} = \frac{d}{dt} \frac{dL(\mathbf{v})}{d\mathbf{v}} = 0 \\ 
 &\therefore \mathbf{v} = \text{constante}.
 \end{align*}


Hay que resaltar dos detalles sobre el lagrangiano para continuar con esta descripción:
\begin{enumerate}

  \item Dos lagrangianos $L$ y $\mathcal{L}$ que difieren bajo una función $f(q,t)$ de la forma $\mathcal{L}(q,\dot{q},t) = L(q,\dot{q},t) + \frac{d f(q,t)}{dt}$ son equivalentes ya que $\delta S_{L} = \delta S_{\mathcal{L}} = 0$\footnote{$ \delta S_{\mathcal{L}} = \cancelto{0}{\delta S_L} + \delta f(q^{(2)},t_2) - \delta f(q^{(1)},t_1) = \pder{f(q^{(2)},t_2)}{q} \cancelto{0}{ \delta q^{(2)} } - \pder{f(q^{(1)},t_1)}{q} \cancelto{0}{ \delta q^{(1)} } = 0$.} y, por tanto, la forma de las ecuaciones definidas por (\ref{eq:lagrange_equations}) no cambia.

 \item Si dos lagrangianos $L_1$ y $L_2$ no interactúan entre sí en un mismo sistema, entonces ambos se pueden englobar en uno mismo como la suma indivudual de sus partes, i.e.
 \begin{equation}
  L = L_1 + L_2 .
  \label{eq:lagrangian_addititivy}
 \end{equation}
 
\end{enumerate}

Por lo primero, podemos encontrar la forma explícita de $L$ para la partícula libre. Sabemos que en el marco $\mathcal{K}$ la partícula se mueve con velocidad $\mathbf{v}$. Si otro marco inercial $\mathcal{K}'$ se mueve a $\mathbf{\epsilon} \ll \mathbf{v}$ respecto a $\mathcal{K}$, entonces la partícula se con velocidad $\mathbf{v}' = \mathbf{v} + \mathbf{\epsilon}$ en $\mathcal{K}'$. Así,
\begin{align*}
 \mathcal{L} = L(v'^2) = L(v^2 + 2 \mathbf{v} \cdot \mathbf{\epsilon} + \epsilon^2) \\
 \implies  L(v'^2) = L(v^2) + \pder{L}{v^2} 2 \mathbf{v} \cdot \mathbf{\epsilon} + \cancelto{0}{\mathcal{O}(\epsilon^2)}.
\end{align*}
Como $\mathcal{L}$ y $L$ sólo pueden diferir por la derivada temporal de alguna función $f(\mathbf{v},t)$, entonces
\begin{align*}
 \pder{L}{v^2} = \alpha
\end{align*}
y, por tanto
\begin{equation}
 L(v^2) = \alpha v^2 = \frac{1}{2}m v^2,
\end{equation}
con $m$ la masa de la partícula, consiguiendo que el lagrangiano tenga unidades de energía.

Por lo segundo, como las partículas libres, por definición, no interactúan entre sí, el lagrangiano de un sistema de $N$ partículas libres será
\begin{equation}
 L(\mathbf{v}_1, \ldots, \mathbf{v}_N) = L_1(\mathbf{v}_1) + \ldots + L_N(\mathbf{v}_N) = \sum_{i=1}^N \frac{1}{2}m_i v_i^2.
 \label{eq:free_part_lagrangian}
\end{equation}