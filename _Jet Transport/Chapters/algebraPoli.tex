%% ************************************************************************
%% &&&&&&&&&&&&&&&&   01. ÁLGEBRA POLINOMIAL  &&&&&&&&&&&&&&&&&&&&&&&&&&&&&
\section{Construcción del Álgebra}

Como se menciona en la introducción, para poder hacer el transporte alrededor de todo una vecindad $\mathcal{U}$ de manera numérica es importante diseñar un integrador de las ecuaciones que cargue toda la información de $\mathcal{U}$ para cada paso temporal. Esto se puede lograr si en lugar de hacer evolucionar al integrador con escalares en $\mathbb{R}$ o $\mathbb{C}$ se hace con un polinomio con centro en la condición inicial $\bm{x_0} := \bm{x}(t=0)$ , de modo que para cada paso se tiene un nuevo polinomio que representa la evolución del polinomio en el paso anterior, el cual se puede sustituir por reales y obtener así puntos de la vecindad de $\bm{x_0}$. \\

Dicho lo anterior, es necesario desarrollar un álgebra polinomial que denotaremos como $\algpol$ donde $\pk$ es el conjunto de \textbf{polinomios de orden $n$ con coeficientes en el campo $\mathbb{K}$} tal que, si  $P(x) \in \pk$, éste se define como
$$P = P(x) := \polk{p_k} $$
con $P: \mathbb{C} \to \mathbb{K}$ una función analítica y $p_k \in \mathbb{K} \  \forall i \in [0,n]$.

%		- It may be necessary to analize how different independet variables imply that A(x) != A(b).


Inspirado en la aritmética usual en $\mathbb{C}$ o $\mathbb{R}$, podemos definir $(+,\cdot)$ para $\algpol$ de la siguiente manera:

Sean $A,B \in \pk$ con $A = \polk{a_k}$ y $B = \polk{b_k} \ a_k,b_k \in \mathbb{K} \ \forall k \in [0,n]$, entonces, para la suma, notemos que
\begin{align*}
\polk{a_k} + \polk{b_k} =& a_0 + a_1 x + \cdots a_n x^n + b_0 + b_1 x + \cdots + b_n x^n \\
=& (a_0 + b_0) + (a_1 + b_1) x + \cdots + (a_n + b_n)x^n = \polk{(a_n + b_n)} 
\end{align*}
así, se define \textbf{$+$}  en $\pk$ como
\begin{equation}
A + B = \polk{(a_k + b_k)}.
\label{eq: polsum}
\end{equation}

Para el producto, notemos que
\begin{align*}
\polk{a_k}\cdot\polk{b_k} =& (a_0 + a_1 x + \cdots + a_n x^n)\cdot(b_0 + b_1 x + \cdots + b_n x^n) \\
=& (a_0b_0) + (a_1b_0 + a_0b_1) x + (a_2b_0 + a_1b_1 + a_0b_2) x^2 + \cdots + \\
+& \sum_{j=0}^n a_{n-j}b_j x^n + \mathcal{O}(x^{n+1}) = \polk{ \polj{a_{k-j}b_j } }
\end{align*}

así, se define \textbf{$\cdot$}  en $\pk$ como

\begin{equation}
A \cdot B = \polk{\sum_{j=0}^k a_{k-j}b_j}.
\label{eq: polprod}
\end{equation}
Aún cuando las operaciones están definidas inspiradas en la aritmética de los números, $\mathbb{K}$ puede ser cualquier campo arbitrario.

\begin{proposicion}
Para $\mathbb{K} = \mathbb{R}$ o $\mathbb{C}$, $\algpol$ forma un campo.
\end{proposicion}

\begin{proof}
Basta probar las nueve propiedades de campo con las operaciones de \ref{eq: polsum} y \ref{eq: polprod}.
Sean $A$, $B$ y $C \in \pk$
\begin{enumerate}

 \item $ A + B = B + A $
 \begin{proof}
  \begin{align*}
   A + B =& \polk{a_k} + \polk{b_k}  = \polk{(a_k + b_k)} \\ 
   =& \polk{(b_k + a_k)} = \polk{b_k} +     \polk{a_k} = B + A
  \end{align*}
 \end{proof}
 
 \item A + (B+C) = (A+B) + C
 \begin{proof}
  \begin{align*}
   A + (B+C) =& \polk{a} + \polk{(b_k+c_k)} = \polk{(a_k + (b_k + c_k))} \\
   =&  \polk{((a_k + b_k) + c_k)} = \polk{(a_k+b_k)} + \polk{c_k} \\
   =& (A+B) + C 
  \end{align*}
 \end{proof}
 
 \item $\exists \  \bm{0} \in \pk \text{ tal que }  \bm{0} + A = A $
 \begin{proof}
 Sea $\mathbb{0} = \bm{0}(x) = \polk{\sigma_k}$ donde $\sigma_k = 0 \ \forall k \in [0,n]$, así 
  \begin{align*}
  \bm{0} + A = \polk{(\sigma_k + a_k)} = \polk{(0 + a_k)} = \polk{a_k} = A
  \end{align*}
 \end{proof}
 
 \item $\exists \ {-A} \in\pk  \text{ tal que } A + (-A) = 0 $
 \begin{proof}
 Sea $-A = -A(x) = \polk{\mathrm{a}_k}$ donde $\mathrm{a}_k = -a_k \ \forall k \in [0,n]$, así
  \begin{align*}
   A + (-A) = \polk{(a_k + \mathrm{a}_k)} = \polk{(a_k + (-a_k))} = \polk{0} = \bm{0}
  \end{align*}
 \end{proof}
 
 \item $ A\cdot B = B\cdot A $
 \begin{proof}
 Hay que probar, básicamente, que $\sum_{j=0}^k{a_{k-j}b_j} = \sum_{j=0}^k{b_{k-j}a_j}$:
  \begin{align*} 
   \sum_{j=0}^k{a_{k-j}b_j} =& \sum_{i=0}^k{a_ib_{k-i}} \text{ con } i = k - j \\
   =& \sum_{i=0}^k{ b_{k-i}a_i } = \sum_{j=0}^k { b_{k-j}a_j }
  \end{align*}
  $\therefore A\cdot B = B\cdot A$.
 \end{proof}
 
 \item $ (A \cdot B) \cdot C = A \cdot (B \cdot C) $
 \begin{proof}
 Por un lado
  \begin{align*}
  A \cdot (B \cdot C) =& \polk{a_k}\ \polk{ \polj{b_{k-j}c_j} } = \polk{\big(  \sum_{j=0}^ka_{k-j}\sum_{i=0}^jb_{j-i}c_i \big)} \\
  =& \polk{\big(  \sum_{r+j=k}a_r\sum_{p+i=j}b_pc_i \big)} = \polk{\big(  \sum_{r+j=k}\sum_{p+i=j}a_rb_pc_i \big)} \\
  =& \polk{\big(  \sum_{r+p+i=k}a_rb_pc_i \big)} 
  \end{align*}
  por otro
    \begin{align*}
  (A \cdot B) \cdot C =& \polk{ \polj{a_{k-j}b_j} }\ \polk{c_k} = \polk{\big( \sum_{i=0}^ja_{j-i}b_i  \sum_{j=0}^kc_{k-j} \big)} \\
  =& \polk{\big(  \sum_{r+j=k}(\sum_{p+i=r}a_pb_i) c_j \big)} = \polk{\big(  \sum_{r+j=k}\sum_{p+i=r}a_pb_ic_j \big)} \\
  =& \polk{\big(  \sum_{p+i+j=k}a_pb_ic_j \big)} 
  \end{align*}
  
  Basta ver que, por un cambio de nombre de índices $p \to r$, $i \to p$, $j \to i$,
  \begin{align*}
   \sum_{p+i+j=k}a_pb_ic_j = \sum_{r+p+i=k}a_rb_pc_i
  \end{align*}   
  $\therefore (A \cdot B) \cdot C = A \cdot (B \cdot C)$.
 \end{proof}
 
 \item $\exists \ \textbf{1} \in \pk \text{ tal que } \textbf{1}\cdot A = A $
 \begin{proof}
 Sea $ \textbf{1} = \textbf{1}(x) = \polk{\omega_k}$ donde $\omega_k = 0 \ \forall k > 0$ y $\omega_0 = 1$, así:
 \begin{align*}
  \textbf{1} \cdot A = \polk{ \polj{a_{k-j}\omega_j} } = \polk{ a_{k-0} \cdot 1 } 
  =& \polk{a_k} = A 
 \end{align*}
 \end{proof}
 
 \item $\exists \  A^{-1} \in \pk \text{ tal que } A \cdot A^{-1} = \textbf{1} \ \forall A \text{ con } a_o \neq 0 $
 \begin{proof}
 Sea $A^{-1} = A^{-1}(x) = \polk{\alpha_k}$; con 
 \end{proof}
 \begin{align*}
  A \cdot A^{-1} = \polk{ \polj{a_{k-j}\alpha_j} } \text{ así, buscando} \\
  \sum_{j=0}^k a_{k-j} \alpha_j = 0 \ \forall \ k>0 \text{ y }  \alpha_0 = \frac{1}{a_0}  
 \end{align*}
 podemos llegar a una fórmula recursiva para $\alpha_k$
 \begin{align*}
 \alpha_k = -\frac{1}{a_0} \sum_{j=0}^{k-1}a_{k-j}\alpha_j.
 \end{align*}
 Así, por construcción, se obtiene que $ A \cdot A^{-1} = \textbf{1} $.
 \item $ A \cdot (B+C) = A \cdot B + A \cdot C $
 \begin{proof}
  \begin{align*}
  A \cdot (B+C) =& \polk{a_k} \big(\polk{(b_k + c_k)} \big) = \polk{\polj{a_{k-j}(b_j+c_j)}} \\
  =& \polk{ \big( \polj{a_{k-j}b_j} \polj{a_{k-j}c_j} \big) } = A \cdot B + A \cdot C
  \end{align*}   
 \end{proof}
 
\end{enumerate}
Así, quedan demostradas todas las propiedades de campo. \qed
\end{proof}

Visto eso, es práctico definir otras operaciones que serán de ayuda al trabajar con elementos de $\pk$. Serán la potencia, la composición y la derivada nuestros ` caballos de batalla '' para el desarrollo de casi cualquier función existente en la aritmética usual.

Sea $A = A(x) = \polk{a_k} \in \pk$. Si $m \in \mathbb{N}$,
\begin{align*}
 A^m =& \big(\polk{a_k}\big)^m = \polk{a_k}\polk{a_k}\big(\polk{a_k}\big)^{m-2} \\
 =& \polk{\polj{a_{k-j}a_j}}\polk{a_k}\big(\polk{a_k}\big)^{m-3} = \polk{{^1\alpha_k}}\polk{a_k}\big(\polk{a_k}\big)^{l-3} \\
 =& \cdots = \polk{{^m\alpha_k}} \\
 &\text{con } {^1\alpha_k} = \polj{a_{k-j}a_j} \text{ y } {^m\alpha_k} = \polj{{^{m-1}\alpha_k}a_j} \\
\end{align*}
y , aunque no queda clara la forma explícita de ${^m\alpha_k}$, se muestra que la \textbf{potencia} está bien definida y que $A^m \in \pk$.


Con esto último, y sea $B = \polk{b_k} \in \pk$, se puede desarrollar
\begin{align*}
 B(A(x)) = B(A) =& \sum_{k=0}^n b_k \big( \sum_{j=0}^n a_j x^j  \big)^k = \polk{b_k} \sum_{j=0}^n {^k\alpha_j x^j} =\polk{ \polj{b_{k-j} {^k\alpha_j} } }
 \end{align*}
que corresponde a la \textbf{composición} de $A$ en $B$, la cual está bien definida y pertenece a $\pk$.

Inspirados en la derivada usual de polinomios de orden $n$ $P: \mathbb{C} \to \mathbb{C}$, tenemos que

\begin{align*}
 \frac{dP(x)}{dx} := P'(x) = P' = \sum_{k=0}^n{k p_k x^{k-1}} = \polk{(k+1)p_{k+1}} 
\end{align*}

así, se define la \textbf{derivada} de $A$ como

\begin{align}
 A' = \polk{(k+1)a_{k+1}}
 \label{eq:polderiv}
\end{align}

con $a_{n+1} := 0 \in \mathbb{K}$ tal que $A' \in \pk$. De la definición usual de derivada podemos ver cómo la \textbf{regla de la cadena} también tiene sentido en este campo, ya que 
\begin{align*}
 { A(B(x))}' :=& \lim_{x \to a}\frac{A(B(x)) - A(B(a))}{x-a} = \lim_{x \to a} \frac{A(B(x)) - A(B(a))}{B(x)-B(a)}\frac{B(x) - B(a)}{x-a}  \\
 =& \lim_{x \to a} \frac{A(B(x)) - A(B(a))}{B(x)-B(a)} \lim_{x \to a} \frac{B(x) - B(a)}{x-a} = A(B(x))'B(x)'
\end{align*}

lo cual se puede traducir a 
\begin{align*}
\lim_{x \to a} \big( A(B(x)) + \ - A(B(a)) \big) \cdot \big( (B(x) + \ -B(a) \big)^{-1} \cdot \lim_{x \to a} \big( (B(x) + \ -B(a) \big) \cdot \big( \textbf{1(x)} + \ -\textbf{1(a)} \big)    
\end{align*}
con las operaciones definidas hasta ahora.

%		- See "chain rule" technical details. Maybe only mention taht is has sense for polynomial functions and later on prove that it is indeed true.
%		- Define (/,-,sin,cos,exp,log,pow). This should be expanded when needed in the thesis only.

Aún cuando ya tenemos las operaciones básicas del álgebra definidas, va a ser de gran practicidad extender otras operaciones útiles para el transporte de $\mathbb{U}$ para el campo vectorial que estemos estudiando.

Se define la \textbf{resta $-$} como
\begin{equation}
 A - B := A + (-B).
 \label{eq:polisub}
\end{equation}


Para la división, sea $D(x) \in \pk$, tal que  $B \cdot D = A $,  entonces 
\begin{align*}
 & \polk{ \polj{b_{k-j} d_j} } = \polk{ a_k } \implies \polk{ a_k - \polj{b_{k-j} d_j} } = 0 \\ 
 &\implies a_k - ( \sum_{j=0}^{k-1}a_{k-j} d_j + b_0 d_k ) = 0 \\
 & \therefore d_k = \frac{1}{b_0} \big( a_k - \sum_{j=0}^{k-1} b_{k-j} d_j \big)  
\end{align*}

Se define la \textbf{división} como
\begin{equation}
 D = A/B \text{ con } d_k = \frac{1}{b_0} \big( a_k - \sum_{j=0}^{k-1} b_{k-j} d_j \big)
 \label{eq:polidiv}
\end{equation}

Con el mismo espíritu, inspirados por el artículo de Haro ~\cite{Haro2009} podemos definir más relaciones de recurrencia para otras operaciones entre polinomios.

%		- Maybe I can make just a few and then cite "Haro"... in the spirit of Haro, other functions will be defined as ---

Fantástico, $^nP_{\mathbb{C}}$ y $^nP_{\mathbb{R}}$ son campos y tenemos ya varias operaciones definidas sobre $\pk$, pero la motivación del desarrollo de este álgebra es hacer polinomios de una variable (que representen al tiempo) cuyos coeficientes sean polinomios de varias variables (que representen al espacio fase de $\dot{x} = f(x(t))$), cuyos coeficientes sean elementos de $\mathbb{C}$ o $\mathbb{R}$. Es decir, nos gustaría trabajar con polinomios cuyos coeficientes sean polinomios, cuyos coeficientes sean, finalmente, números.
Sea  $\pk$ con $\mathbb{K} = {{^{n}P_{\mathbb{C}}}}$, entonces, con $P(x) \in \pk$ y $P_k(y) = \sum_{j=0}^n a_{j,k}y^j \in \mathbb{K} \forall k \in [1,n]$

\begin{equation}
 P(x) =  \polk{P_k(y)} = \polk{\left( \sum_{j=0}^n a_{j,k} y^j \right)} := \sum_{k+j=0}^n a_{j,k} y^j x^k = P(\mathbf{x}).
\label{eq:pkn2}
\end{equation}

Con la definición planteada en la última igualdad de \ref{eq:pkn2} se observa cómo el polinomio $P(x)$ cuyos coeficientes son los polinomios $P_k(y)$ es equivalente a un polinomio de dos variables que vive en un espacio que denominaremos, por construcción, $\pkn{2}$. Se tuvo que imponer la condición de que en el último conjunto de términos $i+j = n$ así $P(x)$  pueda pertenecer, en efecto, a un espacio de polinomios de orden $n$.

Inductivamente, se puede construir el espacio $\pkn{N}$ tomando
\begin{align}
 P(x_1) = \sum_{k_{1}=0}^n\sum_{k_2=0}^n\cdots\sum_{k_N=0}^n a_{k_{1},k_{2},\cdots,k_{N}}x_N^{k_{N}}x_{N-1}^{k_{N-1}} \cdots x_1^{k_{1}} := \sum_{||\mathbf{k}||_{1}=0}^n a_{\mathbf{k}}\mathbf{x}^{\mathbf{k}}
\label{eq:pknN}
\end{align}

con $\mathbf{k} = (k_1,\cdots,k_N)$, $||\mathbf{k}||_1 = k_1+k_2+\cdots+k_N$ la 1-norma de $\mathbf{k}$ y $\mathbf{x} = (x_1,\cdots,x_N)^T$. Así, quedan bien definidos los polinómios de orden $n$ en $N$ variables con coeficientes en $\mathbb{C}$ (y en $\mathbb{R}$, naturalmente) en el espacio $\pkn{N}$ como la suma de polinomios homogeneos desde orden $0$ hasta $N$.

%%Con esto contruido, se puede hablar de los Taylors anidados de manera muy natural, que nacen de NO truncar a orden n el espacio; notar que los espacios $\pk$ con $\mathbb{K} = {{^{n}P_{\mathbb{C}}}}$ y $\pkn{2}$ son escencialmente distintos ya que no son del mismo orden...

En el caso del transporte de jets se trabaja en el espacio ${^{m}P_{\mathbb{K}}}$, $\mathbb{K} = \pkn{N}$, con $m$ el orden de la expansión temporal del método de Taylor y $n$ el orden de la expansión polinomial de $f$ en \ref{eq:ode} evaluada en $\U_{\xo}$. 

\section{Integrador de Taylor}

Ahora que ya están definidas todas las operaciones de polinomios es buen momento para desarrollar al integrador numérico que nos permitirá hacer el transporte mencionadas en la introducción.

La idea básica de un integrador es que, dado un sistema de $n$ ecuaciones diferenciales

\begin{align}
 \dot{\bm{x}(t)} := \frac{dx_i(t)}{dt} = f_i(t;\bm{x}(t)) \ \forall i \in [1,n]
 \label{eq: eqdif}
\end{align}

se encuentre un método que evolucione de la mejor manera posible dada una condición inicial $\bm{x_0}$ o, en nuestro caso, una \textit{bola} inicial {$B_{\bm{x_0}}$} a cada paso del parámetro $t$ (tiempo) que rige la ecuación. El tema con un integrador numérico es que éste pretende evaluar todos los tiempos contenidos entre la condición inicial y la final y, como ninguna máquina ha sido capaz de evaluar una cantidad continua en un intervalo, es necesario discretizar al parámetro que hace que nuestras ecuaciones evolucionen.

Una forma muy intuitiva de abordar este problema es con el método de Euler, que parte de la definición usual de derivada. Tomemos el caso de una función analítica $f: \mathbb{R}  \to \mathbb{R}$, entonces

\begin{align}
 \dot{f}(t) := \frac{df(t)}{dt} = \lim_{h\to 0} \frac{ f(t+h) - f(t) }{h}
 \label{eq: deriv}
\end{align}

Si tomamos el límite $h \to 0$ de una manera menos rigurosa y lo tratamos simplemente como un \textit{incremento muy pequeño}, podemos tener una aproximación razonablemente buena de la derivada, donde

\begin{align*}
\dot{f}(t) := \frac{ f(t+h) - f(t) }{h} + R(t;h)
\end{align*}

donde $R$ es un residuo que depende de $h$ y decrece con ella. Si esto lo aplicamos a \ref{eq: eqdif} cuando $\textbf{x} \in \bm{R}^n$, entonces

\begin{align}
 \bm{f}(\bm{x}(t)) = \frac{d\bm{x}(t)}{dt} =& \frac{ \bm{x}(t+h) - \bm{x}(t) }{h}+ \bm{R}(h)\\
 \implies \bm{x}(t+h) =& \bm{x}(t) + h\cdot \bm{f}(\bm{x}(t)) - \bm{R}(t;h)
 \label{eq: poli1g}
\end{align}
\begin{align}
 \to \ & x_{i_{n+1}} = x_{i_n} + h\cdot f_n - R_{1_i}(h) \ \forall i \in [1,n]
 \label{eq: euler}
\end{align}

que plantea una relación de recurrencia, conocida como el \textbf{método de Euler}. Aún cuando esta relación no es tan poderosa y no es, definitivamente, la que se usará en los cálculos desarrollados en la presente tesis, es interesante notar que \ref{eq: euler} es ahora un \textit{mapeo} que resuelve la ecuación diferencial de manera discreta, con un \textit{salto temporal} de tamaño $h$ y un error o residuo $R_{1_i}(h)$ y esto se conservará para cualquier método empleado.\\

En lugar de tomar una condición inicial $\bm{x_0}$, nos interesa tomar el polinomio

\begin{align*}
 \ball{0} = \bm{x}_0 + \bm{\mu}
\end{align*}

que representa a la condición inicial $\bm{x_0}$ más alguna perturbación definida  por el vector de polinomios $\bm{\mu}$, similar a como se hace en \cite{Perez2015}. Con las operaciones definidas en la sección anterior se puede desarrollar la relación de recurrencia \ref{eq: euler} tal que 

\begin{align*}
 \bm{x_1} =&  \ball{1} =  \ball{0} + h\cdot\bm{f}( \ball{0}) \\ 
 =& \big( \bm{x_0} + h\cdot \bm{f}(\bm{x_0}) \big) +  \big( \bm{\mu} + h\cdot \bm{f}(\bm{\mu}) \big) + \bm{R}_1(h)
\end{align*}

que representa el primer paso temporal del método de Euler pero con la condición inicial $\ball{0}$, que hace uso del álgebra de polinomios desarrollado en la sección anterior. Es interesante notar que el tranporte de bolas si puede hacer con cualquier integrador numérico. Generalizar este método es simplemente extender la o las funciones del sistema de ecuaciones diferenciales asociadas hasta algún orden $N$, donde si $f_i$ en \ref{eq: eqdif} es analítica $\forall i$, entonces 

\begin{align}
 \bm{x}(t+h) = \bm{x}(t) + \bm{x}^{\prime}(t)h + \frac{1}{2!}\bm{x}^{\prime\prime}(t)h^2 + ... + \frac{1}{n!}\bm{x}^{(n)}(t)h^n + \bm{R}^{n+1}(h)
\end{align}
que, por \ref{eq: eqdif},
\begin{align}
 \bm{x}(t+h) =& \bm{x}(t) + \bm{f}(t;\bm{x}(t))h + \frac{1}{2!}\bm{f}^\prime(t;\bm{x}(t))h^2 \\ 
 +& ... + \frac{1}{n!}\bm{f}^{(n-1)}(t;\bm{x}(t))h^n + \bm{R}^{n+1}(h)
 \label{eq: taylorint}
\end{align}

definiendo así la relación de recurrencia del \textbf{método de Taylor}. Notemos que \ref{eq: taylorint} también puede ser evaluada por $\ball{0}$, siempre y cuando estén definidas las operaciones para $\bm{f}$ y todas sus derivadas en el intervalo de interés.

Hay que prestar especial atención al error que pueda tener la ecuación en cada paso temporal, ya que éste determinará qué tan preciso es el método y qué tan fiable es la solución que obtenemos al integrar para un intervalo de tiempo dado. Se busca encontrar cómo varía el error en función del orden del polinomio escogido y el tamaño del paso de integración $h$.

En el apéndice \textit{Taylor series methods of integration} de Simó (2001) ~\cite{Simo2001} se propone y prueba que para obtener obtener un error relativo predeterminado $\epsilon$ en cada paso, el paso de integración más eficiente es 
\begin{equation}
 h = \frac{ \rho}{e^{2}}
\end{equation}

cuando el error escogido $\epsilon$ tiende a $0$, con $\rho = \rho(t)$ el radio de convergencia de la serie asociada a la función desarrollada. Así, para el método de Taylor tradicional podemos proponer, como se hizo en \cite{Perez2015}, un paso de integración dependiente del estado actual del sistema dinámico

\begin{align}
 h := \min{ \left\lbrace \left(\frac{ \epsilon e^2 \norm{\bm{f}} }{ \norm{\bm{f}^{(n-2)}}}\right)^{  \frac{1}{n-2} } , \left(\frac{ \epsilon e^2 \norm{\bm{f}}}{ \norm{\bm{f}^{(n-1)}}}\right)^{ \frac{1}{n-1} }  \right\rbrace }
\end{align}

donde se toman los últimos dos coeficientes de la serie en caso de que la última divergiese. Nótese que se toma el error relativo aunque hay ocasiones donde el error absoluto es preferido.

Ahora, si \ref{eq: taylorint} es evaluada en bolas, entonces se debe tomar en consideración que cada uno de los coeficientes  de la serie es un vector de polinomios de tantas variables como los grados de libertad de la ecuación diferencial. En este caso, los coeficientes de la j-ésima variable $f_j^(i)$ para su i-ésima derivada se representan como $f_j^{(i)} = \sum_k f_{j,k}^{(i)} \eta^k$. Para que tenga sentido hablar de un paso óptimo, se propone en \cite{Perez2015} tomar 

\begin{align*}
 \psi_k^{(i)} := \max_{j \in [1,N]}{|f_{j,k}^{(i)}|}
\end{align*}

que representa la dirección de mayor crecimiento del k-ésimo coeficiente de la i-ésima derivada. Si el orden al que expandimos las series es $L$, esto nos genera un conjunto de $L \ \psi_k^{(i)}$  para cada $i$, de modo que se plantea 

\begin{align*}
 h := \min_{k \in [1,L]}{ \left\lbrace \left(\frac{ \epsilon e^2 \psi_k^{(1)} }{ \psi_k^{(n-2)}}\right)^{  \frac{1}{n-2} } , \left(\frac{ \epsilon e^2 \psi_k^{(1)}}{ \psi_k^{(n-1)}}\right)^{ \frac{1}{n-1} }  \right\rbrace }
\end{align*}

como el nuevo paso óptimo adaptado al álgebra polinomial que se ha desarrollado. Generalmente, si los coeficientes de la serie decrecen con el orden de esta, las $k$s mínimas serán siempre cercanas a $L$, por la convergencia que se presupone de las funciones del sistema dinámico.

%		- ¿En qué unidades trabajar? ...puede ser interesante incluir esa discusión, posiblemente pertenezca al siguiente capítulo (jet transport).
%		- Releer todo a ver si todo es coherente.
