%% ************************************************************************
%% &&&&&&&&&&&&&&&&   01. ÁLGEBRA POLINOMIAL  &&&&&&&&&&&&&&&&&&&&&&&&&&&&&

Como se menciona en la introducción, para poder hacer el transporte alrededor de todo una vecindad $\mathcal{U}$ de manera numérica es importante diseñar un integrador de las ecuaciones que cargue toda la información de $\mathcal{U}$ para cada paso temporal. Esto se puede lograr si en lugar de hacer evolucionar al integrador con escalares en $\mathbb{R}$ o $\mathbb{C}$ se hace con un polinomio con centro en la condición inicial $\mathbf{x_0} := \mathbf{x}(t=0)$ , de modo que para cada paso se tiene un nuevo polinomio que representa la evolución del polinomio en el paso anterior, el cual se puede sustituir por reales y obtener así puntos de la vecindad de $\mathbf{x_0}$. \\

Dicho lo anterior, es necesario desarrollar un álgebra polinomial que denotaremos como $\mathcal{A}(^{n}P_{\mathbb{K}},+,\cdot)$ donde $^{n}P_{\mathbb{K}}$ es el conjunto de \textbf{polinomios de orden $n$ con coeficientes en el campo $\mathbb{K}$} tal que, si  $P(x) \in \pk$, éste se define como
$$P = P(x) := \polk{p_k} $$
con $p_k \in \mathbb{K} \  \forall i \in [0,n]$.

%Aquí desarrollar operaciones (+,·). Procurar hacerlo general para cualquier campo K.

\begin{proposicion}
Para $\mathbb{K} = \mathbb{R}$ o $\mathbb{C}$, $\mathcal{A}(^{n}P_{\mathbb{K}},+,\cdot)$ forma un campo.
\end{proposicion}

\begin{proof}
Basta probar las nueve propiedades de campo con las operaciones 1asdasd y 2asdasd definidas; iremos una por una.
Sean $A$, $B$ y $C \in \pk$
\begin{enumerate}

 \item $ A + B = B + A $
 \begin{proof}
 %%to0 large
  \begin{align*}
   A + B =& \polk{a_k} + \polk{b_k}  = \polk{(a_k + b_k)} \\ 
   =& \polk{(b_k + a_k)} = \polk{b_k} +     \polk{a_k} = B + A
  \end{align*}
 \end{proof}
 
 \item A + (B+C) = (A+B) + C
 \begin{proof}
 %too large
  \begin{align*}
   A + (B+C) =& \polk{a} + \polk{(b_k+c_k)} = \polk{(a_k + (b_k + c_k))} \\
   =&  \polk{((a_k + b_k) + c_k)} = \polk{(a_k+b_k)} + \polk{c_k} \\
   =& (A+B) + C 
  \end{align*}
 \end{proof}
 
 \item $\exists \  \mathbf{0} \in \pk \text{ tal que }  \mathbf{0} + A = A $
 \begin{proof}
 Sea $\mathbb{0} = \mathbf{0}(x) = \polk{\sigma_k}$ donde $\sigma_k = 0 \ \forall k \in [0,n]$, así 
  \begin{align*}
  \mathbf{0} + A = \polk{(\sigma_k + a_k)} = \polk{(0 + a_k)} = \polk{a_k} = A
  \end{align*}
 \end{proof}
 
 \item $\exists \ {-A} \in\pk  \text{ tal que } A + (-A) = 0 $
 \begin{proof}
 Sea $-A = -A(x) = \polk{\mathrm{a}_k}$ donde $\mathrm{a}_k = -a_k \ \forall k \in [0,n]$, así
  \begin{align*}
   A + (-A) = \polk{(a_k + \mathrm{a}_k)} = \polk{(a_k + (-a_k))} = \polk{0} = \mathbf{0}
  \end{align*}
 \end{proof}
 
 \item $ A\cdot B = B\cdot A $
 \begin{proof}
 Hay que probar, básicamente, que $\sum_{j=0}^k{a_{k-j}b_j} = \sum_{j=0}^k{b_{k-j}a_j}$:
  \begin{align*} 
   \sum_{j=0}^k{a_{k-j}b_j} =& \sum_{i=0}^k{a_ib_{k-i}} \text{ con } i = k - j \\
   =& \sum_{i=0}^k{ b_{k-i}a_i } = \sum_{j=0}^k { b_{k-j}a_j }
  \end{align*}
  $\therefore A\cdot B = B\cdot A$.
 \end{proof}
 
 \item $ (A \cdot B) \cdot C = A \cdot (B \cdot C) $
 \begin{proof}
 Por un lado
  \begin{align*}
  A \cdot (B \cdot C) =& \polk{a_k}\ \polk{ \polj{b_{k-j}c_j} } = \polk{\big(  \sum_{j=0}^ka_{k-j}\sum_{i=0}^jb_{j-i}c_i \big)} \\
  =& \polk{\big(  \sum_{r+j=k}a_r\sum_{p+i=j}b_pc_i \big)} = \polk{\big(  \sum_{r+j=k}\sum_{p+i=j}a_rb_pc_i \big)} \\
  =& \polk{\big(  \sum_{r+p+i=k}a_rb_pc_i \big)} 
  \end{align*}
  por otro
    \begin{align*}
  (A \cdot B) \cdot C =& \polk{ \polj{a_{k-j}b_j} }\ \polk{c_k} = \polk{\big( \sum_{i=0}^ja_{j-i}b_i  \sum_{j=0}^kc_{k-j} \big)} \\
  =& \polk{\big(  \sum_{r+j=k}(\sum_{p+i=r}a_pb_i) c_j \big)} = \polk{\big(  \sum_{r+j=k}\sum_{p+i=r}a_pb_ic_j \big)} \\
  =& \polk{\big(  \sum_{p+i+j=k}a_pb_ic_j \big)} 
  \end{align*}
  
  Basta ver que, por un cambio de nombre de índices $p \to r$, $i \to p$, $j \to i$,
  \begin{align*}
   \sum_{p+i+j=k}a_pb_ic_j = \sum_{r+p+i=k}a_rb_pc_i
  \end{align*}   
  $\therefore (A \cdot B) \cdot C = A \cdot (B \cdot C)$.
 \end{proof}
 
 \item $\exists \ \textbf{1} \in \pk \text{ tal que } \textbf{1}\cdot A = A $
 \begin{proof}
 Sea $ \textbf{1} = \textbf{1}(x) = \polk{\omega_k}$ donde $\omega_k = 0 \ \forall k > 0$ y $\omega_0 = 1$, así:
 \begin{align*}
  \textbf{1} \cdot A = \polk{ \polj{a_{k-j}\omega_j} } = \polk{ a_{k-0} \cdot 1 } 
  =& \polk{a_k} = A 
 \end{align*}
 \end{proof}
 
 \item $\exists \  A^{-1} \in \pk \text{ tal que } A \cdot A^{-1} = \textbf{1} \ \forall A \text{ con } a_o \neq 0 $
 \begin{proof}
 Sea $A^{-1} = A^{-1}(x) = \polk{\alpha_k}$; con 
 \end{proof}
 \begin{align*}
  A \cdot A^{-1} = \polk{ \polj{a_{k-j}\alpha_j} } \text{ así, buscando} \\
  \sum_{j=0}^k a_{k-j} \alpha_j = 0 \ \forall \ k>0 \text{ y }  \alpha_0 = \frac{1}{a_0}  
 \end{align*}
 podemos llegar a una fórmula recursiva para $\alpha_k$
 \begin{align*}
 \alpha_k = -\frac{1}{a_0} \sum_{j=0}^{k-1}a_{k-j}\alpha_j.
 \end{align*}
 Así, por construcción, se obtiene que $ A \cdot A^{-1} = \textbf{1} $.
 \item $ A \cdot (B+C) = A \cdot B + A \cdot C $
 \begin{proof}
  \begin{align*}
  A \cdot (B+C) =& \polk{a_k} \big(\polk{(b_k + c_k)} \big) = \polk{\polj{a_{k-j}(b_j+c_j)}} \\
  =& \polk{ \big( \polj{a_{k-j}b_j} \polj{a_{k-j}c_j} \big) } = A \cdot B + A \cdot C
  \end{align*}   
 \end{proof}
 
\end{enumerate}
Así, quedan demostradas todas las propiedades de campo. \qed
\end{proof}



%missing: 
%		  - Define operations (+, ·) explicitely before making the proposition.
%		  - Prove that this algebra is also a field over \pk
%		  - Prove, as a corolary, that this algebra is also a vector space over K (optional).
%		  - Definir otras operaciones necesarias para el desarrollo del algebra (/,-,sin,cos,exp,log,pow)


 
